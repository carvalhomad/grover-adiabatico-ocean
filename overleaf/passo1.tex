% ============================================================
% passo1.tex
% Teorema Adiabático — versão consistente em notação de Dirac
% ============================================================

\chapter{O Teorema Adiabático}

\section{Introdução}

O teorema adiabático constitui um dos resultados fundamentais da mecânica quântica, estabelecendo as condições sob as quais o vetor de estado de um sistema quântico fechado permanece associado a um autoestado instantâneo do Hamiltoniano quando este varia lentamente no tempo. Esse resultado é central na formulação da computação quântica adiabática, pois garante que a evolução controlada de um Hamiltoniano permite preparar o autoestado fundamental associado ao Hamiltoniano final \cite{Born1928, Kato1950, Messiah1962}.

Fisicamente, o teorema descreve a evolução de um vetor de estado $\ket{\psi(t)}$ no espaço de Hilbert quando o operador Hamiltoniano $\hat{H}(t)$ depende continuamente do tempo. Quando essa dependência é suficientemente lenta e certas hipóteses espectrais são satisfeitas, o vetor de estado permanece associado ao autoestado instantâneo correspondente.

\section{Formulação matemática}

Considere um sistema quântico fechado descrito por um operador Hamiltoniano dependente do tempo,

\begin{equation}
\hat{H}(t), \qquad t \in [0,T].
\end{equation}

A evolução temporal do vetor de estado $\ket{\psi(t)}$ é governada pela equação de Schrödinger dependente do tempo,

\begin{equation}
i \hbar \frac{d}{dt} \ket{\psi(t)} = \hat{H}(t) \ket{\psi(t)}.
\end{equation}

Para cada instante de tempo, o operador Hamiltoniano admite uma decomposição espectral em termos de seus autoestados instantâneos,

\begin{equation}
\hat{H}(t) \ket{n(t)} = E_n(t) \ket{n(t)},
\end{equation}

onde:

\begin{itemize}

\item $\ket{n(t)}$ representa o autoestado instantâneo associado ao autovalor $E_n(t)$;

\item o conjunto $\{\ket{n(t)}\}$ forma uma base ortonormal completa do espaço de Hilbert;

\item os autovalores são ordenados de forma crescente,

\begin{equation}
E_0(t) < E_1(t) < E_2(t) < \cdots.

\end{equation}

\end{itemize}

O autoestado $\ket{0(t)}$ corresponde ao autoestado fundamental instantâneo.

A decomposição espectral do Hamiltoniano pode ser escrita como

\begin{equation}
\hat{H}(t) =
\sum_n E_n(t) \ket{n(t)} \bra{n(t)}.
\end{equation}

\section{Hipóteses formais do teorema adiabático}

O teorema adiabático depende das seguintes hipóteses formais \cite{Kato1950, Jansen2007}:

\begin{enumerate}

\item O sistema é fechado e sua evolução é descrita por um operador unitário $\hat{U}(t)$ tal que

\begin{equation}
\ket{\psi(t)} = \hat{U}(t)\ket{\psi(0)}.
\end{equation}

\item O operador Hamiltoniano $\hat{H}(t)$ é suficientemente regular no tempo, sendo pelo menos continuamente diferenciável.

\item Existe uma lacuna espectral finita entre o autoestado fundamental e o primeiro autoestado excitado, definida por

\begin{equation}
\Delta(t) =
E_1(t) - E_0(t) > 0.
\end{equation}

\item A evolução ocorre em uma escala de tempo suficientemente longa.

\end{enumerate}

Define-se a lacuna espectral mínima como

\begin{equation}
\Delta_{\min} =
\min_{t \in [0,T]}
\left(
E_1(t) - E_0(t)
\right).
\end{equation}

\section{Enunciado do teorema adiabático}

Considere que o sistema seja inicialmente preparado no autoestado fundamental do Hamiltoniano inicial,

\begin{equation}
\ket{\psi(0)} = \ket{0(0)}.
\end{equation}

Sob as hipóteses estabelecidas, o teorema adiabático afirma que, para evolução suficientemente lenta, o vetor de estado em um instante posterior $t$ satisfaz

\begin{equation}
\ket{\psi(t)} =
e^{i \theta(t)}
\ket{0(t)}
+
\ket{\epsilon(t)},
\end{equation}

onde:

\begin{equation}
\lim_{T \to \infty} \ket{\epsilon(t)} = \ket{0},
\end{equation}

ou seja, o vetor de estado permanece arbitrariamente próximo do autoestado fundamental instantâneo.

A fase total $\theta(t)$ pode ser decomposta como

\begin{equation}
\theta(t) =
\theta_{\mathrm{dyn}}(t)
+
\theta_{\mathrm{geom}}(t),
\end{equation}

onde

\begin{equation}
\theta_{\mathrm{dyn}}(t)
=
-
\frac{1}{\hbar}
\int_0^t
E_0(t') dt'
\end{equation}

é a fase dinâmica, e

\begin{equation}
\theta_{\mathrm{geom}}(t)
=
i \int_0^t
\bra{0(t')}
\frac{d}{dt'}
\ket{0(t')}
dt'
\end{equation}

é a fase geométrica \cite{Berry1984}.

\section{Condição adiabática}

Uma condição quantitativa suficiente para a validade da evolução adiabática pode ser expressa como

\begin{equation}
\max_{t}
\frac{
\left|
\bra{1(t)}
\frac{d\hat{H}(t)}{dt}
\ket{0(t)}
\right|
}{
\Delta(t)^2
}
\ll 1.
\end{equation}

Essa condição estabelece que a taxa de variação do Hamiltoniano deve ser pequena em comparação com o quadrado da lacuna espectral.

\section{Interpretação em termos de projetores}

Define-se o projetor sobre o autoestado fundamental como

\begin{equation}
\hat{P}_0(t) =
\ket{0(t)} \bra{0(t)}.
\end{equation}

O teorema adiabático afirma que, para evolução suficientemente lenta,

\begin{equation}
\ket{\psi(t)}
\approx
e^{i\theta(t)}
\hat{P}_0(t)
\ket{\psi(t)}.
\end{equation}

Isso significa que o vetor de estado permanece no subespaço fundamental instantâneo.

\section{Interpretação física}

Fisicamente, o vetor de estado acompanha continuamente o autoestado fundamental instantâneo do Hamiltoniano. A presença de uma lacuna espectral finita impede transições para autoestados excitados durante a evolução lenta.

Esse princípio constitui o fundamento da computação quântica adiabática, na qual a solução de um problema é codificada no autoestado fundamental de um Hamiltoniano final \cite{Farhi2000, Albash2018}.

\section{Implicações para computação quântica}

Na computação quântica adiabática, define-se um Hamiltoniano dependente de um parâmetro adimensional $s$,

\begin{equation}
\hat{H}(s),
\end{equation}

com:

\begin{equation}
\hat{H}(0) = \hat{H}_0,
\qquad
\hat{H}(1) = \hat{H}_f,
\end{equation}

onde $\ket{0(0)}$ é facilmente preparável e $\ket{0(1)}$ codifica a solução do problema.

O teorema adiabático garante que a evolução suficientemente lenta transforma

\begin{equation}
\ket{0(0)}
\longrightarrow
\ket{0(1)}.
\end{equation}

Esse resultado estabelece a base teórica da computação quântica adiabática.