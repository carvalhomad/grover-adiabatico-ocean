% ============================================================
% passo2.tex
% Modelo de Computação Quântica Adiabática
% ============================================================

\chapter{O Modelo de Computação Quântica Adiabática}

\section{Introdução}

A computação quântica adiabática constitui um modelo de computação quântica baseado na evolução contínua e controlada de um sistema físico descrito por um Hamiltoniano dependente do tempo. Esse modelo fundamenta-se diretamente no teorema adiabático da mecânica quântica, discutido no capítulo anterior, que garante a preservação do estado fundamental de um sistema quando seu Hamiltoniano varia suficientemente lentamente \cite{Farhi2000, Albash2018}.

Diferentemente do modelo tradicional de computação quântica baseado em circuitos, no qual a computação é realizada por meio da aplicação discreta de portas quânticas, o modelo adiabático utiliza uma evolução contínua do Hamiltoniano para transformar um estado inicial conhecido em um estado final que codifica a solução de um problema computacional \cite{Aharonov2004}.

\section{Formulação matemática do modelo adiabático}

Considere um sistema quântico fechado cujo Hamiltoniano depende de um parâmetro adimensional $s$, definido no intervalo:

\begin{equation}
s \in [0,1].
\end{equation}

O Hamiltoniano do sistema é definido como uma interpolação entre um Hamiltoniano inicial $H_0$ e um Hamiltoniano final $H_f$:

\begin{equation}
H(s) = (1 - s) H_0 + s H_f.
\end{equation}

O Hamiltoniano inicial $H_0$ é escolhido de modo que seu estado fundamental seja facilmente preparável, enquanto o Hamiltoniano final $H_f$ é construído de forma que seu estado fundamental codifique a solução do problema computacional de interesse \cite{Farhi2000}.

A dependência temporal é introduzida definindo-se:

\begin{equation}
s = \frac{t}{T},
\end{equation}

onde $T$ representa o tempo total de evolução.

O Hamiltoniano dependente do tempo torna-se, portanto:

\begin{equation}
H(t) = H\left( \frac{t}{T} \right).
\end{equation}

A evolução do estado do sistema é governada pela equação de Schrödinger:

\begin{equation}
i \hbar \frac{d}{dt} \ket{\psi(t)} = H(t) \ket{\psi(t)}.
\end{equation}

Se o sistema for inicialmente preparado no estado fundamental de $H_0$,

\begin{equation}
\ket{\psi(0)} = \ket{0(0)},
\end{equation}

o teorema adiabático garante que, para evolução suficientemente lenta, o sistema permanecerá no estado fundamental instantâneo e, ao final da evolução, atingirá o estado fundamental do Hamiltoniano final:

\begin{equation}
\ket{\psi(T)} \approx \ket{0(1)}.
\end{equation}

Esse estado contém a solução do problema computacional.

\section{Codificação de problemas computacionais}

O princípio fundamental da computação quântica adiabática consiste em codificar o problema computacional no Hamiltoniano final.

Considere um problema de otimização cuja função custo é definida por:

\begin{equation}
C(x),
\end{equation}

onde $x$ representa uma configuração do sistema.

O Hamiltoniano final é construído de forma que:

\begin{equation}
H_f \ket{x} = C(x) \ket{x}.
\end{equation}

O estado fundamental corresponde à configuração que minimiza a função custo.

Dessa forma, a computação adiabática transforma o problema computacional em um problema físico de minimização de energia.

\section{Tempo de execução e lacuna espectral}

O tempo necessário para que a evolução seja adiabática depende da lacuna espectral mínima do Hamiltoniano interpolado:

\begin{equation}
\Delta_{\min} = \min_{s \in [0,1]} \left( E_1(s) - E_0(s) \right).
\end{equation}

O tempo total de evolução deve satisfazer aproximadamente a condição \cite{Jansen2007, Albash2018}:

\begin{equation}
T \gg \frac{1}{\Delta_{\min}^2}.
\end{equation}

Essa relação estabelece um vínculo direto entre a complexidade computacional e a estrutura espectral do Hamiltoniano.

Se a lacuna espectral diminuir exponencialmente com o tamanho do sistema, o tempo de execução também crescerá exponencialmente, limitando a eficiência do algoritmo.

\section{Equivalência com o modelo de circuitos}

Um resultado fundamental na teoria da computação quântica é a demonstração da equivalência entre o modelo de computação quântica adiabática e o modelo baseado em circuitos quânticos \cite{Aharonov2004}.

Esse resultado estabelece que qualquer algoritmo implementável no modelo de circuitos pode ser implementado no modelo adiabático com overhead polinomial, e vice-versa.

Essa equivalência garante que a computação quântica adiabática possui o mesmo poder computacional que o modelo padrão de computação quântica.

\section{Implementação física em sistemas reais}

A computação quântica adiabática pode ser implementada fisicamente utilizando sistemas quânticos cujo Hamiltoniano pode ser controlado experimentalmente.

Um exemplo importante é o modelo baseado em sistemas Ising, cujo Hamiltoniano é dado por:

\begin{equation}
H = \sum_i h_i \sigma_i^z + \sum_{i<j} J_{ij} \sigma_i^z \sigma_j^z,
\end{equation}

onde $\sigma_i^z$ representa operadores de Pauli e $h_i$ e $J_{ij}$ representam campos locais e acoplamentos entre qubits.

O Hamiltoniano completo implementado em sistemas reais inclui também um termo de campo transverso:

\begin{equation}
H(s) = A(s) \sum_i \sigma_i^x + B(s) H_f,
\end{equation}

onde o termo proporcional a $\sigma_i^x$ atua como Hamiltoniano inicial, e o termo proporcional a $H_f$ representa o Hamiltoniano problema.

\section{Limitações físicas e tecnológicas}

Embora o modelo adiabático seja teoricamente poderoso, sua implementação física apresenta desafios significativos.

Entre os principais fatores limitantes estão:

\begin{itemize}

\item Decoerência, resultante da interação com o ambiente;

\item Ruído térmico;

\item Imperfeições no controle experimental do Hamiltoniano;

\item Pequenas lacunas espectrais, que exigem tempos de evolução muito longos;

\item Escalabilidade limitada dos sistemas físicos disponíveis.

\end{itemize}

Esses fatores podem induzir transições não adiabáticas e reduzir a probabilidade de obtenção do estado fundamental correto.

\section{Aplicações}

A computação quântica adiabática possui aplicações em diversas áreas, incluindo:

\begin{itemize}

\item Problemas de otimização combinatória;

\item Problemas de busca;

\item Aprendizado de máquina;

\item Física de sistemas fortemente correlacionados;

\item Modelagem de materiais e sistemas moleculares.

\end{itemize}

Essas aplicações exploram a capacidade do sistema quântico de explorar eficientemente o espaço de estados.

\section{Relevância tecnológica}

Sistemas físicos baseados em computação quântica adiabática já foram construídos experimentalmente, incluindo dispositivos baseados em qubits supercondutores.

Esses sistemas permitem implementar Hamiltonianos Ising programáveis e executar algoritmos de otimização por meio de evolução adiabática controlada \cite{Albash2018}.

Embora ainda existam limitações tecnológicas, esses dispositivos representam uma das implementações mais avançadas de computação quântica disponíveis atualmente.

