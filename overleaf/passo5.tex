% ============================================================
% passo5.tex
% Simulação do algoritmo de Grover adiabático usando Ocean SDK
% ============================================================

\chapter{Simulação do Algoritmo de Grover Adiabático Utilizando o D-Wave Ocean SDK}

\section{Introdução}

A implementação prática do algoritmo de Grover em sua formulação adiabática requer a simulação da evolução de um sistema quântico governado por um Hamiltoniano dependente do tempo. Embora a implementação física completa envolva dispositivos quânticos especializados, como processadores quânticos baseados em qubits supercondutores, é possível modelar e simular o Hamiltoniano final utilizando ferramentas computacionais clássicas.

Neste capítulo, será apresentada a simulação do Hamiltoniano final do algoritmo de busca adiabático utilizando o pacote D-Wave Ocean SDK. O objetivo consiste em modelar o Hamiltoniano problema associado ao estado alvo e verificar a obtenção do estado fundamental correspondente, que representa a solução do problema de busca.

\section{Formulação do Hamiltoniano problema}

Conforme discutido no capítulo anterior, o Hamiltoniano final do algoritmo de busca adiabático é definido como:

\begin{equation}
H_f = I - \ket{m}\bra{m},
\end{equation}

onde $\ket{m}$ representa o estado marcado.

No contexto da implementação computacional utilizando o Ocean SDK, esse Hamiltoniano deve ser representado na forma de um Hamiltoniano de Ising, definido como:

\begin{equation}
H =
\sum_i h_i \sigma_i^z +
\sum_{i<j} J_{ij} \sigma_i^z \sigma_j^z,
\end{equation}

onde:

\begin{itemize}

\item $h_i$ representa o campo aplicado ao qubit $i$;

\item $J_{ij}$ representa o acoplamento entre os qubits $i$ e $j$;

\item $\sigma_i^z$ representa o operador de Pauli correspondente ao qubit $i$.

\end{itemize}

O estado fundamental desse Hamiltoniano corresponde à configuração de spins que minimiza a energia.

\section{Representação do estado alvo}

Para um sistema composto por $n$ qubits, o espaço de estados possui dimensão:

\begin{equation}
N = 2^n.
\end{equation}

Neste trabalho, foi considerado um sistema composto por oito qubits, resultando em um espaço de busca com:

\begin{equation}
N = 256.
\end{equation}

O estado alvo corresponde ao número inteiro 3, cuja representação binária é:

\begin{equation}
00000011.
\end{equation}

Utilizando a correspondência entre variáveis binárias e variáveis de spin:

\begin{equation}
0 \rightarrow -1, \quad 1 \rightarrow +1,
\end{equation}

obtém-se a configuração de spins correspondente ao estado alvo.

\section{Construção do Hamiltoniano de Ising}

O Hamiltoniano foi construído de forma que o estado alvo corresponda ao estado de menor energia. Isso foi obtido definindo-se os coeficientes do Hamiltoniano como:

\begin{equation}
h_i = - s_i,
\end{equation}

onde $s_i$ representa o valor do spin correspondente ao estado alvo.

Essa construção garante que o estado alvo minimize a energia do sistema.

Os termos de acoplamento foram definidos como:

\begin{equation}
J_{ij} = 0.
\end{equation}

Essa escolha simplifica o Hamiltoniano e mantém o estado alvo como estado fundamental.

\section{Ferramenta de simulação}

A simulação foi realizada utilizando o D-Wave Ocean SDK, um conjunto de bibliotecas que permite modelar e resolver problemas utilizando Hamiltonianos do tipo Ising \cite{Albash2018}.

Foi utilizado o sampler clássico denominado:

\begin{equation}
\texttt{SimulatedAnnealingSampler}
\end{equation}

Esse algoritmo utiliza recozimento simulado clássico para encontrar o estado de menor energia do Hamiltoniano.

Embora esse método não reproduza a dinâmica quântica completa, ele permite identificar o estado fundamental do Hamiltoniano problema.

\section{Procedimento de simulação}

O procedimento adotado consistiu nas seguintes etapas:

\begin{enumerate}

\item Definição do número de qubits;

\item Construção da representação do estado alvo;

\item Construção dos coeficientes do Hamiltoniano de Ising;

\item Construção do modelo quadrático binário correspondente;

\item Execução do algoritmo de recozimento simulado;

\item Identificação do estado de menor energia obtido.

\end{enumerate}

O número de execuções independentes foi definido como:

\begin{equation}
\texttt{num\_reads} = 1000,
\end{equation}

permitindo avaliar a consistência dos resultados.

\section{Resultados obtidos}

A simulação produziu um conjunto de estados candidatos, cada um associado a um valor de energia.

O estado correspondente à menor energia foi identificado como:

\begin{equation}
00000011,
\end{equation}

correspondendo exatamente ao estado alvo definido.

Esse resultado confirma que o Hamiltoniano foi corretamente construído e que o estado fundamental corresponde à solução do problema de busca.

\section{Interpretação dos resultados}

O resultado obtido demonstra que o modelo de Hamiltoniano utilizado codifica corretamente o problema de busca.

O estado fundamental do Hamiltoniano corresponde ao estado alvo, conforme esperado pela formulação teórica do algoritmo de busca adiabático.

Embora a simulação tenha sido realizada utilizando um algoritmo clássico, o resultado confirma a validade da construção do Hamiltoniano problema.

\section{Limitações da simulação}

É importante destacar que o recozimento simulado clássico não reproduz a dinâmica quântica completa descrita pelo modelo adiabático.

Em particular, essa simulação não modela:

\begin{itemize}

\item Evolução unitária quântica;

\item Tunelamento quântico;

\item Lacuna espectral do Hamiltoniano interpolado;

\item Transições adiabáticas reais.

\end{itemize}

Apesar dessas limitações, a simulação permite validar a construção do Hamiltoniano problema.

