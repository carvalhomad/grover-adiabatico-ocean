% ============================================================
% passo7.tex
% Comparação entre o algoritmo de Grover no modelo de circuitos
% e no modelo de computação quântica adiabática
% ============================================================

\chapter{Comparação entre o Algoritmo de Grover no Modelo de Circuitos e no Modelo Adiabático}

\section{Introdução}

O algoritmo de Grover representa um dos exemplos mais importantes de aceleração quântica em relação à computação clássica, reduzindo o tempo necessário para busca não estruturada de $O(N)$ para $O(\sqrt{N})$ \cite{Grover1996}. Esse algoritmo foi inicialmente formulado no modelo de computação quântica baseado em circuitos, mas posteriormente foi demonstrado que o mesmo problema pode ser resolvido utilizando computação quântica adiabática \cite{Farhi1998, Roland2002}.

A equivalência entre essas duas abordagens constitui um resultado fundamental na teoria da computação quântica e fornece uma compreensão mais profunda da natureza física da aceleração quântica.

Neste capítulo, será apresentada uma comparação detalhada entre essas duas implementações, analisando seus fundamentos matemáticos, interpretação física, vantagens, desvantagens e aplicações.

\section{Modelo de circuitos quânticos}

No modelo de circuitos quânticos, a computação é realizada por meio da aplicação sequencial de operadores unitários, denominados portas quânticas.

O algoritmo de Grover inicia com a preparação do estado de superposição uniforme:

\begin{equation}
\ket{\psi_0} =
\frac{1}{\sqrt{N}} \sum_{x=0}^{N-1} \ket{x}.
\end{equation}

O algoritmo utiliza dois operadores fundamentais:

\begin{itemize}

\item O operador oráculo:

\begin{equation}
U_f = I - 2 \ket{m}\bra{m},
\end{equation}

que inverte a fase do estado marcado;

\item O operador de difusão:

\begin{equation}
U_s = 2 \ket{\psi_0}\bra{\psi_0} - I,
\end{equation}

que amplifica a amplitude do estado marcado.

\end{itemize}

A aplicação repetida desses operadores resulta em uma rotação no espaço de Hilbert, amplificando progressivamente a probabilidade de medir o estado solução.

O número ótimo de iterações é dado por:

\begin{equation}
O(\sqrt{N}).
\end{equation}

\section{Modelo de computação quântica adiabática}

No modelo adiabático, a computação é realizada por meio da evolução contínua de um Hamiltoniano dependente do tempo.

O Hamiltoniano interpolado é definido como:

\begin{equation}
H(s) = (1-s) H_0 + s H_f,
\end{equation}

onde:

\begin{equation}
H_0 = I - \ket{\psi_0}\bra{\psi_0}
\end{equation}

e

\begin{equation}
H_f = I - \ket{m}\bra{m}.
\end{equation}

O sistema é inicialmente preparado no estado fundamental de $H_0$ e evolui até atingir o estado fundamental de $H_f$, que corresponde à solução do problema.

Roland e Cerf demonstraram que, utilizando um schedule adequado, o tempo de evolução necessário é:

\begin{equation}
T = O(\sqrt{N}),
\end{equation}

correspondendo exatamente à complexidade do algoritmo de Grover \cite{Roland2002}.

\section{Equivalência matemática}

Ambos os algoritmos operam efetivamente em um subespaço bidimensional gerado pelos estados:

\begin{equation}
\ket{m}
\end{equation}

e

\begin{equation}
\ket{m^\perp}.
\end{equation}

No modelo de circuitos, a evolução ocorre por meio de rotações discretas nesse subespaço.

No modelo adiabático, a evolução ocorre por meio de uma rotação contínua governada pelo Hamiltoniano dependente do tempo.

Essa equivalência geométrica estabelece que ambos os algoritmos exploram a mesma estrutura matemática subjacente.

\section{Equivalência computacional}

Foi demonstrado que nenhum algoritmo quântico pode resolver o problema de busca não estruturada em tempo inferior a $\Omega(\sqrt{N})$ \cite{Bennett1997}.

Como ambas as implementações atingem essa complexidade, conclui-se que ambas são assintoticamente ótimas.

Além disso, foi demonstrado que a computação quântica adiabática é computacionalmente equivalente ao modelo de circuitos quânticos \cite{Aharonov2004}.

Isso implica que qualquer algoritmo implementável em um modelo pode ser implementado no outro.

\section{Diferenças físicas}

Embora matematicamente equivalentes, os dois modelos possuem implementações físicas distintas.

No modelo de circuitos:

\begin{itemize}

\item A computação é realizada por meio de portas discretas;

\item O controle experimental requer a aplicação precisa de pulsos;

\item O algoritmo é implementado como uma sequência de operações unitárias.

\end{itemize}

No modelo adiabático:

\begin{itemize}

\item A computação é realizada por meio da evolução contínua de um Hamiltoniano;

\item O controle experimental envolve a variação gradual de parâmetros físicos;

\item A computação ocorre como um processo físico contínuo.

\end{itemize}

\section{Vantagens do modelo de circuitos}

Entre as vantagens do modelo de circuitos, destacam-se:

\begin{itemize}

\item Maior controle sobre operações individuais;

\item Maior flexibilidade na implementação de algoritmos;

\item Melhor compatibilidade com algoritmos universais.

\end{itemize}

Esse modelo constitui o paradigma dominante na computação quântica teórica.

\section{Vantagens do modelo adiabático}

Entre as vantagens do modelo adiabático, destacam-se:

\begin{itemize}

\item Implementação física mais natural em certos sistemas;

\item Maior robustez a certos tipos de erros;

\item Aplicabilidade direta a problemas de otimização.

\end{itemize}

Essas características tornam o modelo adiabático particularmente relevante para aplicações práticas.

\section{Desvantagens e limitações}

Ambos os modelos possuem limitações.

No modelo de circuitos:

\begin{itemize}

\item Necessidade de controle extremamente preciso;

\item Sensibilidade à decoerência;

\item Complexidade experimental elevada.

\end{itemize}

No modelo adiabático:

\begin{itemize}

\item Dependência crítica da lacuna espectral;

\item Possibilidade de evolução não adiabática;

\item Limitações tecnológicas na implementação física.

\end{itemize}

\section{Cenários de aplicação}

O modelo de circuitos é mais adequado para algoritmos universais, incluindo:

\begin{itemize}

\item Algoritmo de Shor;

\item Algoritmos de simulação quântica;

\item Algoritmos baseados em transformadas quânticas.

\end{itemize}

O modelo adiabático é particularmente adequado para:

\begin{itemize}

\item Problemas de otimização combinatória;

\item Problemas de busca;

\item Problemas de minimização de energia.

\end{itemize}

